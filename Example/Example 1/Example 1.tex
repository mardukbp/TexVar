\documentclass{article}
%
\usepackage{luacode}
\usepackage[fleqn]{amsmath}
% create commands for units (not needed but good practice) 
\newcommand{\msKpW}{\tfrac{m^2K}{W}}
\newcommand{\WpmsK}{\tfrac{W}{m^2K}}
\newcommand{\WpmK}{\tfrac{W}{mK}}
\newcommand{\m}{m}
%
\begin{document}
Calculating the U-Value for an element with two layers and resistance of surface internal and external.\\

\begin{luacode}
	-- load tVar library
	require("tVar.lua")

	-- global Definitions
	tVar.outputMode = "RES_EQ_N"
	tVar.numFormat = "%.2f"
	tVar.numeration = true
		
	-- define variables
	tex.print("\\noindent Resistance of surface")
	R_se = tVar:New(0.04,"R_{se}"):setUnit("\\msKpW"):outRES()
	R_si = tVar:New(0.13,"R_{si}"):setUnit("\\msKpW"):outRES()
	
	tex.print("Parameters for elements")
	d_1 = tVar:New(0.2,"d_1"):setUnit("\\m"):outRES()
	lambda_1 = tVar:New(0.035,"\\lambda_1"):setUnit("\\WpmK"):outRES()
	
	d_2 = tVar:New(0.1,"d_2"):setUnit("\\m"):outRES()
	lambda_2 = tVar:New(0.5,"\\lambda_2"):setUnit("\\WpmK"):outRES()
	
	tex.print("Calculate thermal resistance")
	R = (R_se + d_1/lambda_1 + d_2/lambda_2 + R_si):setName("R"):setUnit("\\msKpW"):print():clean()
	
	tex.print("Calculate U-Value")
	U=(1/R):setName("U"):setUnit("\\WpmsK"):print()
\end{luacode}
\end{document}