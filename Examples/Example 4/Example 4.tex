\documentclass{article}
%
\usepackage{luacode}
\usepackage[fleqn]{amsmath}
\usepackage{amssymb}
%
\begin{document}
	Introducing the \emph{EasyInput} function.\\
	You can send multiple \emph{EasyInput} styled commands to the interpret function.\\\\
	\verb|#Some text| $\rightarrow$ \verb|tex.print("Somet ext")|\\
	\verb|a_k_1_0:=10| $\rightarrow$  \verb|a_k_1_0=tVar:New(10,"a_{k,1}^0")|\\
	(Also possible with vectors and matrices)\\
	\verb|--Some comment| $\rightarrow$  \verb|--Some comment|\\
	\verb|everything else| $\rightarrow$  \verb|everything else| \\
	
	The String hast to be passed as multiline string to the tVar function:
	\begin{verbatim}
	tVar.intString([===[
		EasyInput Code
	]===])
	\end{verbatim}
\begin{luacode}
require("tVar/init.lua")
tVar.intString([===[

#Vector matrix multiplication
-- Global Definitions
tVar.outputMode = "RES_EQ"
tVar.decimalSeparator = ","
tMat.eqTexAsMatrix = true
tVar.qOutput = true

-- Variables
M:={{1,12,23},{421,5,6},{733,81,9}}
v:={4,5,6}

-- Calculations
d=(M*v):setName("d"):print():clean()

#Calculate inverse

ans=(M:Inv()*d):setName(""):print()

]===])	
\end{luacode}
\end{document}